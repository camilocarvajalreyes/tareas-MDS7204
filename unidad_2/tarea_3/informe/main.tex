% Template:     Template Reporte LaTeX
% Documento:    Archivo principal
% Versión:      1.2.9 (22/04/2020)
% Codificación: UTF-8
%
% Autor: Pablo Pizarro R.
%        Facultad de Ciencias Físicas y Matemáticas
%        Universidad de Chile
%        pablo@ppizarror.com
%
% Sitio web:    [https://latex.ppizarror.com/reporte]
% Licencia MIT: [https://opensource.org/licenses/MIT]

% CREACIÓN DEL DOCUMENTO
\documentclass[letterpaper,11pt,oneside]{article}

% INFORMACIÓN DEL DOCUMENTO
\def\titulodelreporte {\textbf{Tarea 3 - Procesos Gaussianos}}
\def\temaatratar {Tarea 3}
\def\fechadelreporte {\today}

\def\autordeldocumento {Camilo Carvajal Reyes}
\def\nombredelcurso {Aprendizaje de Máquinas Avanzado}
\def\codigodelcurso {MDS7204}

\def\nombreuniversidad {Universidad de Chile}
\def\nombrefacultad {Facultad de Ciencias Físicas y Matemáticas}
\def\departamentouniversidad {Departamento de Ingeniería Matemática}
\def\imagendepartamento {departamentos/dim}
\def\localizacionuniversidad {Santiago, Chile}

% CONFIGURACIONES
\input{lib/config}

% IMPORTACIÓN DE LIBRERÍAS
\input{lib/env/imports}

% IMPORTACIÓN DE FUNCIONES Y ENTORNOS
\input{lib/cmd/all}

% IMPORTACIÓN DE ESTILOS
\input{lib/style/all}

% CONFIGURACIÓN INICIAL DEL DOCUMENTO
\input{lib/cfg/init}

\newtheoremstyle{break}%
    {}{}%
    {\itshape}{}%
    {\bfseries}{}% % Note that final punctuation is omitted.
    {\newline}
    % {\thmname{#1}\thmnumber{ #2}\thmnote{ #3}}
    {\thmname{#1}\thmnote{ #3}}

\theoremstyle{break}
\newtheorem{theorem}{Teorema}[subsection]
\newtheorem{proposition}{Proposición}[subsection]

% INICIO DE LAS PÁGINAS
\begin{document}
	
% CONFIGURACIÓN DE PÁGINA Y ENCABEZADOS
\input{lib/cfg/page}

% CONFIGURACIONES FINALES
\input{lib/cfg/final}

% ======================= INICIO DEL DOCUMENTO =======================

% Título y nombre del autor
\inserttitle

% Resumen o Abstract
En la presenta tarea se presentarán aplicaciones de Procesos Gaussianos usando diferentes Kernels y variando los respectivos parámetros. Se usará la primera serie del conjunto de \href{https://ecg.mit.edu/time-series/}{pulsos sanguineos del MIT}, que parece tener arritmia sinusal respiratoria. Se remueven 30\% de los datos, los cuales se intentan predecir con nuestro proceso ajustado al otro 70\% de los datos. Además se usan las siguientes métricas de evaluación: log-versimilitud negativa (para el conjunto de entrenamiento, i.e., aquella usada para entrenar los hiper-parámetros, y para el conjunto de puntos no-observados) y el error cuadrático medio entre la media y los valores reales (de puntos no observados). Para las métricas en cuestión se usa el posterior computado sobre aquellos puntos de interés.

% Ejemplo, se puede borrar
\section{\textit{Kernel Spectral Mixture}}
Consideramos primeramente el kernel de Mixtura Espectral dado por: 
$$K(x)=\displaystyle \sum^Q_{q=1}\sigma^2_q exp(-2\pi^2l^2_qx^2)cos(2\pi\mu_qx).$$
Para esto, usamos el toolkit \href{https://github.com/GAMES-UChile/The_Art_of_Gaussian_Processes}{gp\_lite}, en la cual este kernel viene implementado. Usar los parámetros por defecto nos da un proceso como el mostrado en la figura \ref{img:untrained}. Vale señalar que variar el dominio temporal da distintos resultados en esta fase (no así cuando se ajustan los parámetros). Acá nos movemos en el dominio de los minutos.

\insertimage[\label{img:untrained}]{img/untrained_gp_SM_post}{scale=0.38}{Posterior para GP sin entrenar}

Notemos que \lipsum[1]. Por otro lado ...

\insertimage[\label{img:trained}]{img/trained_gp_SM_post}{scale=0.38}{Posterior para GP entrenada}

\lipsum[2]

\insertimage[\label{img:untrained}]{img/untrained_gp_periodic_post}{scale=0.38}{Posterior para GP sin entrenar}

Notemos que \lipsum[1]. Por otro lado ...

\insertimage[\label{img:trained}]{img/trained_gp_periodic_post}{scale=0.38}{Posterior para GP entrenada}


% FIN DEL DOCUMENTO
\end{document}
