% Template:     Template Reporte LaTeX
% Documento:    Archivo principal
% Versión:      1.2.9 (22/04/2020)
% Codificación: UTF-8
%
% Autor: Pablo Pizarro R.
%        Facultad de Ciencias Físicas y Matemáticas
%        Universidad de Chile
%        pablo@ppizarror.com
%
% Sitio web:    [https://latex.ppizarror.com/reporte]
% Licencia MIT: [https://opensource.org/licenses/MIT]

% CREACIÓN DEL DOCUMENTO
\documentclass[letterpaper,11pt,oneside]{article}

% INFORMACIÓN DEL DOCUMENTO
\def\titulodelreporte {\textbf{Tarea 1 - Filtros de partículas}}
\def\temaatratar {Tarea 1}
\def\fechadelreporte {\today}

\def\autordeldocumento {Camilo Carvajal Reyes}
\def\nombredelcurso {Aprendizaje de Máquinas Avanzado}
\def\codigodelcurso {MDS7204}

\def\nombreuniversidad {Universidad de Chile}
\def\nombrefacultad {Facultad de Ciencias Físicas y Matemáticas}
\def\departamentouniversidad {Departamento de Ingeniería Matemática}
\def\imagendepartamento {departamentos/dim}
\def\localizacionuniversidad {Santiago, Chile}

% CONFIGURACIONES
\input{lib/config}

% IMPORTACIÓN DE LIBRERÍAS
\input{lib/env/imports}

% IMPORTACIÓN DE FUNCIONES Y ENTORNOS
\input{lib/cmd/all}

% IMPORTACIÓN DE ESTILOS
\input{lib/style/all}

% CONFIGURACIÓN INICIAL DEL DOCUMENTO
\input{lib/cfg/init}

% INICIO DE LAS PÁGINAS
\begin{document}
	
% CONFIGURACIÓN DE PÁGINA Y ENCABEZADOS
\input{lib/cfg/page}

% CONFIGURACIONES FINALES
\input{lib/cfg/final}

% ======================= INICIO DEL DOCUMENTO =======================

% Título y nombre del autor
\inserttitle

% Resumen o Abstract
% \begin{abstract}
% 	\lipsum[11]
% \end{abstract}

% Ejemplo, se puede borrar
\lipsum[1]

\begin{table}
\begin{tabular}{|c|l|}
\hline
\textbf{parámetro}        & \multicolumn{1}{r|}{\textbf{valor}} \\ \hline
A                         & \multicolumn{1}{r|}{0.5488}         \\ \hline
B                         & \multicolumn{1}{r|}{1.0860}         \\ \hline
\textbf{C}                & 0.6027                              \\ \hline
\textbf{C}                & 0.5179                              \\ \hline
\textbf{Posición inicial} & 1.4940                              \\ \hline
\end{tabular}
\end{table}

\insertimage[\label{img:serie_x}]{img/serie_x}{scale=0.5}{Serie $X$}
\insertimage[\label{img:serie_y}]{img/serie_y}{scale=0.5}{Serie $Y$}

\lipsum[2]

% \insertimage[\label{img:x_pred_updt}]{img/x_x_updt_x_pred}{scale=0.5}{Serie X junto con las predicciones del filtro pre y post actualización}
\insertimage[\label{img:x_updt}]{img/x_x_updt}{scale=0.5}{Serie X junto con las predicciones del filtro}

% FIN DEL DOCUMENTO
\end{document}