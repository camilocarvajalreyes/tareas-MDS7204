% Template:     Template Reporte LaTeX
% Documento:    Archivo principal
% Versión:      1.2.9 (22/04/2020)
% Codificación: UTF-8
%
% Autor: Pablo Pizarro R.
%        Facultad de Ciencias Físicas y Matemáticas
%        Universidad de Chile
%        pablo@ppizarror.com
%
% Sitio web:    [https://latex.ppizarror.com/reporte]
% Licencia MIT: [https://opensource.org/licenses/MIT]

% CREACIÓN DEL DOCUMENTO
\documentclass[letterpaper,11pt,oneside]{article}

% INFORMACIÓN DEL DOCUMENTO
\def\titulodelreporte {\textbf{Tarea 2 - Análisis de señales}}
\def\temaatratar {Tarea 1}
\def\fechadelreporte {\today}

\def\autordeldocumento {Camilo Carvajal Reyes}
\def\nombredelcurso {Aprendizaje de Máquinas Avanzado}
\def\codigodelcurso {MDS7204}

\def\nombreuniversidad {Universidad de Chile}
\def\nombrefacultad {Facultad de Ciencias Físicas y Matemáticas}
\def\departamentouniversidad {Departamento de Ingeniería Matemática}
\def\imagendepartamento {departamentos/dim}
\def\localizacionuniversidad {Santiago, Chile}

% CONFIGURACIONES
\input{lib/config}

% IMPORTACIÓN DE LIBRERÍAS
\input{lib/env/imports}

% IMPORTACIÓN DE FUNCIONES Y ENTORNOS
\input{lib/cmd/all}

% IMPORTACIÓN DE ESTILOS
\input{lib/style/all}

% CONFIGURACIÓN INICIAL DEL DOCUMENTO
\input{lib/cfg/init}

% INICIO DE LAS PÁGINAS
\begin{document}
	
% CONFIGURACIÓN DE PÁGINA Y ENCABEZADOS
\input{lib/cfg/page}

% CONFIGURACIONES FINALES
\input{lib/cfg/final}

% ======================= INICIO DEL DOCUMENTO =======================

% Título y nombre del autor
\inserttitle

% Resumen o Abstract

% Ejemplo, se puede borrar
\section{Definiciones de la Densidad Espectral de Potencia (PSD)}
\lipsum[1]

\newpage
\section{Estimación espectral}
\subsection{Periodograma}
\lipsum[2]

\newpage
\subsection{Método paramétrico: Lomb-Scargle}
\lipsum[3]

\newpage
\section{Fast Fourier Transform}
\subsection{Complejidad Teórica}
\lipsum[4]

\newpage
\subsection{Complejidad empírica}
\lipsum[5]

% FIN DEL DOCUMENTO
\end{document}