% Template:     Template Reporte LaTeX
% Documento:    Archivo principal
% Versión:      1.2.9 (22/04/2020)
% Codificación: UTF-8
%
% Autor: Pablo Pizarro R.
%        Facultad de Ciencias Físicas y Matemáticas
%        Universidad de Chile
%        pablo@ppizarror.com
%
% Sitio web:    [https://latex.ppizarror.com/reporte]
% Licencia MIT: [https://opensource.org/licenses/MIT]

% CREACIÓN DEL DOCUMENTO
\documentclass[letterpaper,11pt,oneside]{article}

% INFORMACIÓN DEL DOCUMENTO
\def\titulodelreporte {\textbf{Tarea 2 - Análisis de señales}}
\def\temaatratar {Tarea 1}
\def\fechadelreporte {\today}

\def\autordeldocumento {Camilo Carvajal Reyes}
\def\nombredelcurso {Aprendizaje de Máquinas Avanzado}
\def\codigodelcurso {MDS7204}

\def\nombreuniversidad {Universidad de Chile}
\def\nombrefacultad {Facultad de Ciencias Físicas y Matemáticas}
\def\departamentouniversidad {Departamento de Ingeniería Matemática}
\def\imagendepartamento {departamentos/dim}
\def\localizacionuniversidad {Santiago, Chile}

% CONFIGURACIONES
\input{lib/config}

% IMPORTACIÓN DE LIBRERÍAS
\input{lib/env/imports}

% IMPORTACIÓN DE FUNCIONES Y ENTORNOS
\input{lib/cmd/all}

% IMPORTACIÓN DE ESTILOS
\input{lib/style/all}

% CONFIGURACIÓN INICIAL DEL DOCUMENTO
\input{lib/cfg/init}

\newtheoremstyle{break}%
    {}{}%
    {\itshape}{}%
    {\bfseries}{}% % Note that final punctuation is omitted.
    {\newline}
    % {\thmname{#1}\thmnumber{ #2}\thmnote{ #3}}
    {\thmname{#1}\thmnote{ #3}}

\theoremstyle{break}
\newtheorem{theorem}{Teorema}[subsection]
\newtheorem{proposition}{Proposición}[subsection]

% INICIO DE LAS PÁGINAS
\begin{document}
	
% CONFIGURACIÓN DE PÁGINA Y ENCABEZADOS
\input{lib/cfg/page}

% CONFIGURACIONES FINALES
\input{lib/cfg/final}

% ======================= INICIO DEL DOCUMENTO =======================

% Título y nombre del autor
\inserttitle

% Resumen o Abstract

% Ejemplo, se puede borrar
\section{Definiciones de la Densidad Espectral de Potencia (PSD)}
\begin{proposition}[: Identidad Útil]
    $$\displaystyle \sum^N_{k=1}\sum^N_{l=1} f(k-l)=\sum^{N-1}_{\tau=-N+1}(N-|\tau|)f(\tau)$$
\end{proposition}
\begin{proof}
    Probaremos esto por inducción. Para $N=2$ basta notar que
    \begin{alignat*}{2}
        \sum^N_{k=1}\sum^N_{l=1} f(k-l) & = 2f(0)+f(-1) + f(1) = \sum^1_{\tau=-1}(2-|\tau|)f(\tau) = \sum^{N-1}_{\tau=-N+1}(N-|\tau|)f(\tau)
    \end{alignat*}
    Luego para el paso inductivo usamos que
    % \begin{alignat*}{2}
    %     \sum^{N+1}_{k=1}\sum^{N+1}_{l=1} f(k-l) & = \sum^{N-1}_{\tau=-N+1}(N-|\tau|)f(\tau)+\sum^N_{k=1}f(k-N-1)+\sum^N_{l=1}f(N+1-l)+f(0) = % \sum^N_{\tau=-N}(N+1-|\tau|)f(\tau)
    % \end{alignat*}
    $$ \sum^{N+1}_{k=1}\sum^{N+1}_{l=1} f(k-l) & = \sum^{N-1}_{\tau=-N+1}(N-|\tau|)f(\tau)+\sum^N_{k=1}f(k-N-1)+\sum^N_{l=1}f(N+1-l)+f(0) = \sum^N_{\tau=-N}(N+1-|\tau|)f(\tau) $$
    Donde en la primera igualdad usamos la hipótesis inductiva (para $N$), con lo cual concluimos para $N+1$.
\end{proof}
\begin{proposition}[: Definiciones equivalentes de PSD]
    $$\displaystyle \sum_{k\in\mathbb{Z}}c(k)e^{-j\omega k}=\lim_{N\to\infty}\mathbb{E}(\frac{1}{N}|\sum^N_{k=1}x_ke^{-j\omega k}|^2)$$
    Esto nos da la equivalencia de ambas como definiciones de Densidad Espectral de Potencia.
\end{proposition}
\begin{proof}
    En efecto,
    \begin{alignat*}{2}
        \lim_{N\to\infty}\mathbb{E}(\frac{1}{N}|\sum^N_{k=1}x_ke^{-j\omega k}|^2) & = \lim_{N\to\infty}\frac{1}{N}\sum^N_{k=1}\sum^N_{l=1}\mathbb{E}(x_kx_l^*)e^{-j\omega k + j\omega l} \\
        \color{gray}\text{(gracias a la identidad útil)}\color{black} & = \lim_{N\to\infty} \frac{1}{N}\sum^{N-1}_{\tau-N+1}(N-|\tau|)c(\tau)e^{-j\omega\tau} \hspace{.5cm} \color{gray}\text{(con $f(n)=c(n)e^{-j\omega n}$)} \color{black} \\
        &  = \lim_{N\to\infty}(\sum^{N-1}_{\tau=-N+1}c(\tau)e^{-j\omega\tau}-\frac{1}{N}\sum^{N-1}_{\tau-N+1}|\tau|c(\tau)e^{-j\omega\tau}) = \displaystyle \sum_{k\in\mathbb{Z}}c(k)e^{-j\omega k}
    \end{alignat*}
    Para la última igualdad hemos asumido que $\displaystyle \frac{1}{N}\sum^N_{\tau-N}|\tau||r(\tau)|\to0$ cuando $N\to\infty$, que corresponde a asumir que la sucesión $(c(k))_{k\in\mathbb{N}$ decae con suficiente rapidez. Se concluye la equivalencia deseada.
\end{proof}

\newpage
\section{Estimación espectral}
\subsection{Periodograma}
\lipsum[2]

\newpage
\subsection{Método paramétrico: Lomb-Scargle}
\lipsum[3]

\newpage
\section{Fast Fourier Transform}
\subsection{Complejidad Teórica}
\lipsum[4]

\newpage
\subsection{Complejidad empírica}
En esta subsección comprobaremos la complejidad teórica que se obtuvo en la parte anterior. Se implementaron en \textit{python} ambos métodos mencionados.  % : \textbf{DFT} computando $X_k = \displaystyle \sum^{N-1}_{n=1}x_ne^{-j\frac{kn2\pi}{N}$ para cada $k\in\{1,\dots,N\}$; y \textbf{FFT} que realiza lo anterior pero teniendo en cuenta las simetrías de las transformadas.
Primeramente se computan las transformadas con ambos métodos para las siguientes series: ``coseno'', que corresponde a una suma de cosenos: $\sum^5_{n=1}n\cos(n 2\cdot 10\cdot \pi t)$; ``coseno ruido'' que equivale a lo anterior con un ruido $\sim\mathcal{N}(0,0.5)$ en cada paso; y las series de pulso cardiaco $hr_1$ y $hr_2$ de preguntas anteriores.

\newp Cronometramos la ejecución de ambos algoritmos truncando las series en tamaños multiplos de dos. En la figura \ref{img:complejidad_fft1} vemos los tiempos en cuestión con respecto al largo de las series. Además, ploteamos el promedio para $DFT$ y $FFT$. Como referencia, ploteamos también funciones $M_1 N^2$ y $M_2 N\log(N)$ para $M_1$ y $M_2$ constantes.

\insertimage[\label{img:complejidad_fft1}]{img/complejidad_fft1}{scale=0.48}{Tiempo de ejecución de DFT Y FFT versus largo de input.}

% \insertimagerightline[\label{img:complejidad_fft2}]{img/complejidad_fft2}{0.72}{15}{chao}

Los resultados están dentro de lo esperado, pues se vislumbra claramente que el algoritmo $FFT$ disminuye el tiempo de computo. Para graficar de mejor forma que las complejidades reportadas corresponden a las teóricas graficamos las mismas funciones de referencia junto con los tiempos de ejecución para ``coseno ruido''. La figura \ref{img:complejidad_fft2} nos muestra la dominación asintótica de nuestras funciones de referencia. Con esto se tiene que existen en cada caso las constantes $M$ con las cuales podemos afirmar la pertenencia de ambas complejidades a las familias $O(N^2)$ y $O(N\log(N))$ para DFT y FFT respectivamente.

\insertimage[\label{img:complejidad_fft2}]{img/complejidad_fft2}{scale=0.45}{Tiempo de ejecución de DFT Y FFT versus largo de input (en escala logarítmica).}


% FIN DEL DOCUMENTO
\end{document}